\documentclass[12pt]{article}
\usepackage{pmmeta}
\pmcanonicalname{ExampleOfSchwarzChristoffelTransformation}
\pmcreated{2013-03-22 18:20:27}
\pmmodified{2013-03-22 18:20:27}
\pmowner{pahio}{2872}
\pmmodifier{pahio}{2872}
\pmtitle{example of Schwarz-Christoffel transformation}
\pmrecord{6}{40974}
\pmprivacy{1}
\pmauthor{pahio}{2872}
\pmtype{Example}
\pmcomment{trigger rebuild}
\pmclassification{msc}{31A99}
\pmclassification{msc}{30C20}

\endmetadata

% this is the default PlanetMath preamble.  as your knowledge
% of TeX increases, you will probably want to edit this, but
% it should be fine as is for beginners.

% almost certainly you want these
\usepackage{amssymb}
\usepackage{amsmath}
\usepackage{amsfonts}

% used for TeXing text within eps files
%\usepackage{psfrag}
% need this for including graphics (\includegraphics)
%\usepackage{graphicx}
% for neatly defining theorems and propositions
 \usepackage{amsthm}
% making logically defined graphics
%%%\usepackage{xypic}
\usepackage{pstricks}
\usepackage{pst-plot}

% there are many more packages, add them here as you need them

% define commands here

\theoremstyle{definition}
\newtheorem*{thmplain}{Theorem}
\DeclareMathOperator{\arcosh}{arcosh}

\begin{document}
Using the Schwarz-Christoffel transformation, find a function $f$ which maps \PMlinkname{conformally}{ConformalMapping} the upper half of the $z$-plane onto the domain which located above the red broken line of $w$-plane in the below picture, such that
\begin{align}
f(-1) = i \;\; \mbox{and} \;\; f(1) = 0.
\end{align}

\begin{center}
\begin{pspicture}(-3.5,-1.5)(3.7,3.6)
\psaxes[Dx=10,Dy=10]{->}(0,0)(-3.5,-1.5)(3.5,3.5)
\rput(-0.3,3.5){$y$}
\rput(3.55,-0.3){$x$}
\psline[linecolor=red,linewidth=0.04]{-}(-3.5,1)(0,1)
\psline[linecolor=red,linewidth=0.06]{-}(0,1)(0,0)
\psline[linecolor=red,linewidth=0.042]{-}(0,0)(3.3,0)
\psdots[linecolor=red](0,1)(0,0)
\rput(-2.5,1.25){$l$}
\rput(0.2,0.9){$i$}
\rput(0.2,-0.2){$0$}
\rput(2,2.5){$w$-plane}
\end{pspicture}
\end{center}

The border line of the image domain comes first along $l$ from the left and its argument gets in the point \,$w = i$\, the positive increment $\frac{3}{2}\pi = \pi-k_1\pi$,\, whence\, $k_1 = -\frac{1}{2}$.\, Then in the point\, $w = 0$\, the argument of the line attains the increment $\frac{1}{2}\pi = \pi-k_2\pi$,\, whence\, $k_2 = \frac{1}{2}$.\,Thus we have
$$w = f(z) = c\!\int\!\frac{dz}{(z+1)^{-\frac{1}{2}}(z-1)^\frac{1}{2}}
= c\!\int\!\sqrt{\frac{z+1}{z-1}}\,dz = c\,[\sqrt{z^2-1}+\arcosh{z}]+C.$$
The conditions (1) determine the \PMlinkescapetext{constants} $c$ and $C$.\, Hence we obtain the sought function:
$$f(z) \,=\, \frac{1}{\pi}(\sqrt{z^2-1}+\arcosh{z})$$

%%%%%
%%%%%
\end{document}
