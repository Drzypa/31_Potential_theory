\documentclass[12pt]{article}
\usepackage{pmmeta}
\pmcanonicalname{HarmonicFunction}
\pmcreated{2013-03-22 12:43:46}
\pmmodified{2013-03-22 12:43:46}
\pmowner{mathcam}{2727}
\pmmodifier{mathcam}{2727}
\pmtitle{harmonic function}
\pmrecord{9}{33029}
\pmprivacy{1}
\pmauthor{mathcam}{2727}
\pmtype{Definition}
\pmcomment{trigger rebuild}
\pmclassification{msc}{31C05}
\pmclassification{msc}{31B05}
\pmclassification{msc}{31A05}
\pmclassification{msc}{30F15}
\pmrelated{RadosTheorem}
\pmrelated{SubharmonicAndSuperharmonicFunctions}
\pmrelated{DirichletProblem}
\pmrelated{NeumannProblem}

% this is the default PlanetMath preamble.  as your knowledge
% of TeX increases, you will probably want to edit this, but
% it should be fine as is for beginners.

% almost certainly you want these
\usepackage{amssymb}
\usepackage{amsmath}
\usepackage{amsfonts}

% used for TeXing text within eps files
%\usepackage{psfrag}
% need this for including graphics (\includegraphics)
%\usepackage{graphicx}
% for neatly defining theorems and propositions
%\usepackage{amsthm}
% making logically defined graphics
%%%\usepackage{xypic}

% there are many more packages, add them here as you need them

% define commands here

\newcommand{\Prob}[2]{\mathbb{P}_{#1}\left\{#2\right\}}
\begin{document}
A twice-differentiable real or complex-valued function $f\colon U\to\mathbb{R}$ or $f\colon U\to\mathbb{C}$, where $U\subseteq\mathbb{R}^n$ is some \PMlinkescapetext{domain}, is called \emph{harmonic} if its Laplacian vanishes on $U$, i.e. if $$\Delta f\equiv 0.$$

Any harmonic function $f\colon\mathbb{R}^n\to\mathbb{R}$ or $f\colon\mathbb{R}^n\to\mathbb{C}$ satisfies Liouville's theorem.   Indeed, a holomorphic function \emph{is} harmonic, and a real harmonic function $f\colon U\to\mathbb{R}$, where $U\subseteq\mathbb{R}^2$, is locally the real part of a holomorphic function.  In fact, it is enough that a harmonic function $f$ be \PMlinkescapetext{bounded} below (or above) to conclude that it is \PMlinkescapetext{constant}.
%%%%%
%%%%%
\end{document}
