\documentclass[12pt]{article}
\usepackage{pmmeta}
\pmcanonicalname{NeumannProblem}
\pmcreated{2013-03-22 15:19:59}
\pmmodified{2013-03-22 15:19:59}
\pmowner{dczammit}{9747}
\pmmodifier{dczammit}{9747}
\pmtitle{Neumann problem}
\pmrecord{10}{37147}
\pmprivacy{1}
\pmauthor{dczammit}{9747}
\pmtype{Definition}
\pmcomment{trigger rebuild}
\pmclassification{msc}{31B15}
\pmclassification{msc}{31B05}
\pmclassification{msc}{31A05}
\pmrelated{HarmonicFunction}

\endmetadata

% this is the default PlanetMath preamble.  as your knowledge
% of TeX increases, you will probably want to edit this, but
% it should be fine as is for beginners.

% almost certainly you want these
\usepackage{amssymb}
\usepackage{amsmath}
\usepackage{amsfonts}
\usepackage{amsthm}

\usepackage{mathrsfs}

% used for TeXing text within eps files
%\usepackage{psfrag}
% need this for including graphics (\includegraphics)
%\usepackage{graphicx}
% for neatly defining theorems and propositions
%
% making logically defined graphics
%%%\usepackage{xypic}

% there are many more packages, add them here as you need them

% define commands here

\newcommand{\sR}[0]{\mathbb{R}}
\newcommand{\sC}[0]{\mathbb{C}}
\newcommand{\sN}[0]{\mathbb{N}}
\newcommand{\sZ}[0]{\mathbb{Z}}

 \usepackage{bbm}
 \newcommand{\Z}{\mathbbmss{Z}}
 \newcommand{\C}{\mathbbmss{C}}
 \newcommand{\R}{\mathbbmss{R}}
 \newcommand{\Q}{\mathbbmss{Q}}



\newcommand*{\norm}[1]{\lVert #1 \rVert}
\newcommand*{\abs}[1]{| #1 |}



\newtheorem{thm}{Theorem}
\newtheorem{defn}{Definition}
\newtheorem{prop}{Proposition}
\newtheorem{lemma}{Lemma}
\newtheorem{cor}{Corollary}
\begin{document}
Suppose $\Omega$ is a region of $\sR^n$ and $\partial\Omega$ is the boundary of $\Omega$. 
Further suppose $f$ is a function $f\colon\partial \Omega\to\sC$, and suppose $\frac{\partial}{\partial n}$ corresponds to taking a derivative in a direction normal to the boundary $\partial\Omega$ at any point.  Then the 
\emph{Neumann problem} is to find a function $\phi\colon \Omega\cup \partial \Omega \to\sC$
such that 
\begin{eqnarray*}
\frac{\partial\phi}{\partial n} &=& f,\quad \text{on $\partial \Omega$}, \\
\nabla^2 \phi &=& 0,\quad \text{in $\Omega$}. 
\end{eqnarray*}
Here $\nabla^2$ represents the Laplacian operator and the second condition is that $\phi$ be a harmonic function on $\Omega$. The condition for the existence of a solution $\phi$ of the Neumann problem is that integral of the normal derivative of the function $\phi$, calculated over the entire boundary $\partial\Omega$, vanish. This follows from the identic equation
\begin{align*}
\int_{\partial\Omega}\frac{\partial\phi}{\partial n}d\sigma=
\int_\Omega\nabla\!\cdot\!(\nabla\phi)d\tau=\int_\Omega\nabla^2\phi\,d\tau
\end{align*}
and from the fact that $\nabla^2\phi=0$.

%%%%%
%%%%%
\end{document}
