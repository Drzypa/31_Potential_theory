\documentclass[12pt]{article}
\usepackage{pmmeta}
\pmcanonicalname{SchwarzChristoffelTransformationcircularVersion}
\pmcreated{2013-03-22 16:52:57}
\pmmodified{2013-03-22 16:52:57}
\pmowner{stevecheng}{10074}
\pmmodifier{stevecheng}{10074}
\pmtitle{Schwarz-Christoffel transformation (circular version)}
\pmrecord{8}{39136}
\pmprivacy{1}
\pmauthor{stevecheng}{10074}
\pmtype{Result}
\pmcomment{trigger rebuild}
\pmclassification{msc}{31A99}
\pmclassification{msc}{30C20}

% The standard font packages
\usepackage{amssymb}
\usepackage{amsmath}
\usepackage{amsfonts}

% For neatly defining theorems and definitions
%\usepackage{amsthm}

% Including EPS/PDF graphics (\includegraphics)
\usepackage{graphicx}

% Making matrix-based graphics
%%%\usepackage{xypic}

% Enumeration lists with different styles
%\usepackage{enumerate}

% Set up the theorem environments
%\newtheorem{thm}{Theorem}
%\newtheorem*{thm*}{Theorem}

\providecommand{\defnterm}[1]{\emph{#1}}

% The standard number systems
\newcommand{\complex}{\mathbb{C}}
\newcommand{\real}{\mathbb{R}}
\newcommand{\rat}{\mathbb{Q}}
\newcommand{\nat}{\mathbb{N}}
\newcommand{\intset}{\mathbb{Z}}

% Absolute values and norms
% Normal, wide, and big versions of the delimeters
\providecommand{\abs}[1]{\lvert#1\rvert}
\providecommand{\absW}[1]{\left\lvert#1\right\rvert}
\providecommand{\absB}[1]{\Bigl\lvert#1\Bigr\rvert}
\providecommand{\norm}[1]{\lVert#1\rVert}
\providecommand{\normW}[1]{\left\lVert#1\right\rVert}
\providecommand{\normB}[1]{\Bigl\lVert#1\Bigr\rVert}

% Differentiation operators
\providecommand{\od}[2]{\frac{d #1}{d #2}}
\providecommand{\pd}[2]{\frac{\partial #1}{\partial #2}}
\providecommand{\pdd}[2]{\frac{\partial^2 #1}{\partial #2}}
\providecommand{\ipd}[2]{\partial #1 / \partial #2}

% Differentials on integrals
\newcommand{\dx}{\, dx}
\newcommand{\dt}{\, dt}
\newcommand{\dmu}{\, d\mu}

% Inner products
\providecommand{\ip}[2]{\langle {#1}, {#2} \rangle}

% Calligraphic letters
\newcommand{\sF}{\mathcal{F}}
\newcommand{\sD}{\mathcal{D}}

% Standard spaces
\newcommand{\Hilb}{\mathcal{H}}
\newcommand{\Le}{\mathbf{L}}

% Operators and functions occassionally used in my articles
\DeclareMathOperator{\D}{D}
\DeclareMathOperator{\linspan}{span}
\DeclareMathOperator{\rank}{rank}
\DeclareMathOperator{\lindim}{dim}
\DeclareMathOperator{\sinc}{sinc}

% Probability stuff
\newcommand{\PP}{\mathbb{P}}
\newcommand{\E}{\mathbb{E}}

\begin{document}
\PMlinkescapeword{representation}
\PMlinkescapeword{languages}
\PMlinkescapeword{cut}
\PMlinkescapeword{cuts}
\PMlinkescapeword{formula}



The complex-variables function
\[
f(z) = \int_0^z \prod_{k=1}^n (\zeta - z_k)^{\alpha_k-1} d\zeta\,,
\]
maps the closed unit disc $\overline{D} = \{ \abs{z} \leq 1\}$
in the complex plane conformally onto a polygon
with $n$ sides, interior angles $0 < \alpha_k \pi < 2\pi$,
and vertices $f(z_k)$.  (The polygon is assumed to be not self-intersecting.)
The parameters $z_k$ lie on the unit circle,
and depend, generally in a complicated way, on the length of the sides
of the polygon.

The fractional powers $(\zeta - z_k)^{\alpha_k - 1}$
serve to clamp up an arc of the 
circle into a pointy angle of measure $\alpha_k \pi$.  
Indeed, the proof of the Schwarz-Christoffel formula shows that the function $f$
can be decomposed near $z_k$ as
\[
f(z) = f(z_k) + (z-z_k)^{\alpha_k} g_k(z)\,,
\]
where $g_k$ is an analytic function with $g_k(z_k) \neq 0$.
See Figure \ref{fig:clamp}.

\begin{figure}
\begin{center}
\includegraphics{clamp.eps}
\end{center}
\caption{Mapping in a neighborhood of a boundary point}
\label{fig:clamp}
\end{figure}

Note that the exponent is $\alpha_k$ --- not $\alpha_k/2$ ---
because the neigbourhood of a point $z_k$ in the domain space
looks like a half-disc.
For the same reason, the fractional power used in the formula
is to be a single-valued branch continuous on the half-disc.
Finally, the extra $-1$ exponents that appear in the integral
 representation for $f$
come from the power rule for differentiation.

\subsection{Example: $n=3$}

Figure \ref{fig:triangle} illustrates a mapping
from the disc to a triangle ($n=3$).
The contours are the approximate images, under $f$, of circles
of radius $0 < r \leq 1$.

\begin{figure}[!htb]
\begin{center}
\includegraphics{triangle.eps}
\caption{Image of a Schwarz-Christoffel mapping for a triangle}
\end{center}
\label{fig:triangle}
\end{figure}

\begin{figure}
\begin{center}
\includegraphics{domain.eps}
\end{center}
\caption{Corresponding contours of the domain}
\end{figure}

We describe the method used to compute the figure.
Points in the domain $\overline{D}$ are first parameterized 
as $z = re^{i\theta}$,
with $0 \leq r \leq 1$ and $0 \leq \theta < 2\pi$
ranging over a discrete grid, shown schematically in Figure \ref{fig:circle}.
The integral defining the function $f$ is path-independent,
and a natural choice for the paths are
rays emanating from the origin.
When computing the integrals along each ray,
we exploit the additivity of the complex path integral:
\[
f\bigl((r + \Delta r\bigr) e^{i\theta}) = f(re^{i\theta}) 
+ \int_{re^{i\theta}}^{(r+\Delta r) e^{i\theta}} \prod_{k=1}^n (\zeta - z_k)^{\alpha_k - 1} \, d\zeta\,,
\]
so that $f(z)$ is found by summing a previously-computed value
and a new integral to be computed.  And the new integral
is computed using 32-point Gauss quadrature
after reparameterizing the path with $d\zeta = e^{i\theta} \, dr$.

\begin{figure}
\begin{center}
\includegraphics{parameter.eps}
\end{center}
\caption{Parameterization of the domain for computing $f$}
\label{fig:circle}
\end{figure}


The computation of the integrand
\[
\prod_{k=1}^n (\zeta - z_k)^{\alpha_k - 1} = \exp\Bigl( \sum_{k=1}^n (\alpha_k - 1) \log ( \zeta - z_k) \Bigr)
\]
 is straightforward, though 
we must be careful to respect the branch cuts prescribed above.
The $\log$ function in most computer languages
takes a branch cut on the negative axis.
To get the single-valued branches we need in this situation,
we must instead compute $\zeta \mapsto \log ( \zeta - z_k)$ via the expression
\[
\zeta \mapsto \log iz_k + \log \frac{\zeta - z_k}{i z_k}\,,
\]
where $iz_k$ is the direction of the tangent to the circle at the point $z_k$.

Finally, after having obtained a discrete set of image points $f(z)$
traced along each circle $z = re^{i \theta}$,
the contours in the figure are obtained by interpolating a 
curved B\'ezier spline 
through the image points.

If a triangle is prescribed with the vertex locations, 
it is not immediately obvious what the parameters $z_k$ should be to obtain
that triangle.  In the examples here, we simply avoid this difficulty
by arbitrarily choosing the parameters $z_k = e^{2\pi i (k-1)/n}$ 
to be equally spaced on the unit circle, and hope that nice figures result.

The $\alpha_k$ parameters are easily determined from the angles
of the desired figure; they are, in this example:
\[
\alpha_1 = \tfrac14, \quad \alpha_2 = \tfrac12, \quad \alpha_3 = \tfrac14\,.
\]


\subsection{Example: $n=10$}

Figure \ref{fig:star}
shows an example with $n=10$ points.
The strategy for computing this figure is similar to that
of the triangle.

The parameters for this star are (rounded to four decimal places):
\begin{align*}
\alpha_1 = 0.2422 \,,  \quad \alpha_2 = \alpha_1 = 1.3263\,,  \quad \alpha_3 = \alpha_9 = 0.3026 \,, \\
\alpha_4 = \alpha_8 = 1.3026 \,,  \quad \alpha_5 = \alpha_7 = 0.2754\,,  \quad \alpha_6 = 1.3440 \,.
\end{align*}

\begin{figure}
\begin{center}
\includegraphics{star.eps}
\end{center}
\caption{Image of a Schwarz-Christoffel mapping for a star-shaped figure}
\label{fig:star}
\end{figure}

\subsection{Demonstration computer programs}

\begin{itemize}
\item
\PMlinkexternal{Python source code for producing images of the Schwarz-Christoffel transformation}{http://svn.gold-saucer.org/repos/PlanetMath/SchwarzChristoffelTransformationCircularVersion/schwarz-christoffel.py}
\item
\PMlinkexternal{Python source code for the explanatory diagrams}{http://svn.gold-saucer.org/repos/PlanetMath/SchwarzChristoffelTransformationCircularVersion/explanation.py}
\end{itemize}

\begin{thebibliography}{3}
\bibitem{Ahlfors}
Lars V. Ahlfors. {\it Complex Analysis}, third edition. McGraw-Hill, 1979.
\end{thebibliography}

%%%%%
%%%%%
\end{document}
