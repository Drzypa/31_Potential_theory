\documentclass[12pt]{article}
\usepackage{pmmeta}
\pmcanonicalname{PolarSet}
\pmcreated{2013-03-22 14:29:13}
\pmmodified{2013-03-22 14:29:13}
\pmowner{jirka}{4157}
\pmmodifier{jirka}{4157}
\pmtitle{polar set}
\pmrecord{6}{36020}
\pmprivacy{1}
\pmauthor{jirka}{4157}
\pmtype{Definition}
\pmcomment{trigger rebuild}
\pmclassification{msc}{31C05}
\pmclassification{msc}{31B05}
\pmclassification{msc}{31A05}

\endmetadata

% this is the default PlanetMath preamble.  as your knowledge
% of TeX increases, you will probably want to edit this, but
% it should be fine as is for beginners.

% almost certainly you want these
\usepackage{amssymb}
\usepackage{amsmath}
\usepackage{amsfonts}

% used for TeXing text within eps files
%\usepackage{psfrag}
% need this for including graphics (\includegraphics)
%\usepackage{graphicx}
% for neatly defining theorems and propositions
\usepackage{amsthm}
% making logically defined graphics
%%%\usepackage{xypic}

% there are many more packages, add them here as you need them

% define commands here
\theoremstyle{theorem}
\newtheorem*{thm}{Theorem}
\newtheorem*{lemma}{Lemma}
\newtheorem*{conj}{Conjecture}
\newtheorem*{cor}{Corollary}
\newtheorem*{example}{Example}
\newtheorem*{prop}{Proposition}
\theoremstyle{definition}
\newtheorem*{defn}{Definition}
\begin{document}
\begin{defn}
Let $G \subset {\mathbb{R}}^n$ and let
$f \colon G \to {\mathbb{R}} \cup \{ - \infty \}$ be a subharmonic
function which is not identically $-\infty$.
The set ${\mathcal{P}} := \{ x \in G \mid f(x) = - \infty \}$ is
called a {\em polar set}.
\end{defn}

\begin{prop}
Let $G$ and ${\mathcal{P}}$ be as above and suppose that $g$ is a
continuous
subharmonic function on $G \setminus {\mathcal{P}}$.  Then $g$ is subharmonic
on the entire set $G$.
\end{prop}

The requirement that $g$ is continuous cannot be relaxed.

\begin{prop}
Let $G$ and ${\mathcal{P}}$ be as above.  Then the Lebesgue measure of
${\mathcal{P}}$ is 0.
\end{prop}

\begin{thebibliography}{9}
\bibitem{Krantz:several}
Steven~G.\@ Krantz.
{\em \PMlinkescapetext{Function Theory of Several Complex Variables}},
AMS Chelsea Publishing, Providence, Rhode Island, 1992.
\end{thebibliography}
%%%%%
%%%%%
\end{document}
