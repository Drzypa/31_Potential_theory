\documentclass[12pt]{article}
\usepackage{pmmeta}
\pmcanonicalname{DAlembertian}
\pmcreated{2013-03-22 17:55:18}
\pmmodified{2013-03-22 17:55:18}
\pmowner{invisiblerhino}{19637}
\pmmodifier{invisiblerhino}{19637}
\pmtitle{D'Alembertian}
\pmrecord{8}{40414}
\pmprivacy{1}
\pmauthor{invisiblerhino}{19637}
\pmtype{Definition}
\pmcomment{trigger rebuild}
\pmclassification{msc}{31B15}
\pmclassification{msc}{31B05}
\pmclassification{msc}{26B12}
\pmsynonym{wave operator}{DAlembertian}
\pmsynonym{D'Alembert operator}{DAlembertian}
\pmrelated{Laplacian}

% this is the default PlanetMath preamble.  as your knowledge
% of TeX increases, you will probably want to edit this, but
% it should be fine as is for beginners.

% almost certainly you want these
\usepackage{amssymb}
\usepackage{amsmath}
\usepackage{amsfonts}

% used for TeXing text within eps files
%\usepackage{psfrag}
% need this for including graphics (\includegraphics)
%\usepackage{graphicx}
% for neatly defining theorems and propositions
%\usepackage{amsthm}
% making logically defined graphics
%%%\usepackage{xypic}

% there are many more packages, add them here as you need them

% define commands here

\begin{document}
The D'Alembertian is the equivalent of the Laplacian in Minkowskian geometry. It is given by:
\[
\Box = \nabla^2 - \frac{1}{c^2} \frac{\partial^2}{\partial t^2}
\]
Here we assume a Minkowskian metric of the form $(+, +, +, -)$ as typically seen in special relativity. The connection between the Laplacian in Euclidean space and the D'Alembertian is clearer if we write both operators and their corresponding metric.
\subsection{Laplacian}
\[
\mbox{Metric: } ds^2 = dx^2 + dy^2 + dz^2
\]
\[
\mbox{Operator: } \nabla^2 = \frac{\partial^2}{\partial x^2} + \frac{\partial^2}{\partial y^2} + \frac{\partial^2}{\partial z^2}
\]
\subsection{D'Alembertian}
\[
\mbox{Metric: } ds^2 = dx^2 + dy^2 + dz^2 -cdt^2
\]
\[
\mbox{Operator: } \Box = \frac{\partial^2}{\partial x^2} + \frac{\partial^2}{\partial y^2} + \frac{\partial^2}{\partial z^2} - \frac{1}{c^2}\frac{\partial^2}{\partial t^2}
\]

In both cases we simply differentiate twice with respect to each coordinate in the metric. The D'Alembertian is hence a special case of the generalised Laplacian.
\section{Connection with the wave equation}
The wave equation is given by:
\[
\nabla^2 u = \frac{1}{c^2}\frac{\partial^2}{\partial t^2} u
\]
Factorising in terms of operators, we obtain:
\[
(\nabla^2 - \frac{1}{c^2}\frac{\partial^2}{\partial t^2})u = 0
\]
or
\[
\Box u = 0
\]
Hence the frequent appearance of the D'Alembertian in special relativity and electromagnetic theory.
\section{Alternative notation}
The symbols $\Box$ and $\Box^2$ are both used for the D'Alembertian. Since it is unheard of to square the D'Alembertian, this is not as confusing as it may appear. The symbol for the Laplacian, $\Delta$ or $\nabla^2$, is often used when it is clear that a Minkowski space is being referred to.
\section{Alternative definition}
It is common to define Minkowski space to have the metric $(-, +, +, +)$, in which case the D'Alembertian is simply the negative of that defined above:
\[
\Box = \frac{1}{c^2} \frac{\partial^2}{\partial t^2} -\nabla^2
\]
%%%%%
%%%%%
\end{document}
