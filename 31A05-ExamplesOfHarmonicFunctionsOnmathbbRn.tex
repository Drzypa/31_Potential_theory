\documentclass[12pt]{article}
\usepackage{pmmeta}
\pmcanonicalname{ExamplesOfHarmonicFunctionsOnmathbbRn}
\pmcreated{2013-03-22 12:44:23}
\pmmodified{2013-03-22 12:44:23}
\pmowner{mathwizard}{128}
\pmmodifier{mathwizard}{128}
\pmtitle{examples of harmonic functions on $\mathbb{R}^n$}
\pmrecord{9}{33041}
\pmprivacy{1}
\pmauthor{mathwizard}{128}
\pmtype{Example}
\pmcomment{trigger rebuild}
\pmclassification{msc}{31A05}
\pmclassification{msc}{31B05}

% this is the default PlanetMath preamble.  as your knowledge
% of TeX increases, you will probably want to edit this, but
% it should be fine as is for beginners.

% almost certainly you want these
\usepackage{amssymb}
\usepackage{amsmath}
\usepackage{amsfonts}

% used for TeXing text within eps files
%\usepackage{psfrag}
% need this for including graphics (\includegraphics)
%\usepackage{graphicx}
% for neatly defining theorems and propositions
%\usepackage{amsthm}
% making logically defined graphics
%%%\usepackage{xypic}

% there are many more packages, add them here as you need them

% define commands here

\newcommand{\Prob}[2]{\mathbb{P}_{#1}\left\{#2\right\}}
\newcommand{\norm}[1]{\left\|#1\right\|}
\begin{document}
Some real functions in~$\mathbb{R}^n$ (e.g. any linear function, or any affine function) are obviously harmonic functions.  What are some more interesting harmonic functions?

\begin{itemize}
\item
For~$n\ge 3$, define (on the punctured space~$U=\mathbb{R}^n \setminus \{0\}$) the function~$f(x)=\norm{x}^{2-n}$. Then
$$
\frac{\partial f}{\partial x_i} = (2-n) \frac{x_i}{\norm{x}^n},
$$
and
$$
\frac{\partial^2 f}{{\partial x_i}^2} =
n(n-2)\frac{x_i^2}{\norm{x}^{n+2}} - (n-2)\frac{1}{\norm{x}^n}
$$
Summing over $i=1,...,n$ shows $\Delta f\equiv 0$.
\item
For~$n=2$, define (on the punctured plane~$U=\mathbb{R}^2 \setminus \{0\}$) the function~$f(x,y)=\log(x^2+y^2)$.  Derivation and summing yield~$\Delta f\equiv 0$.
\item
For~$n=1$, the condition $(\Delta f)(x)=f''(x)\equiv 0$ forces~$f$ to be an affine function on every segment; there are no ``interesting'' harmonic functions in one dimension.
\end{itemize}
%%%%%
%%%%%
\end{document}
