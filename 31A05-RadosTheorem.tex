\documentclass[12pt]{article}
\usepackage{pmmeta}
\pmcanonicalname{RadosTheorem}
\pmcreated{2013-03-22 14:08:07}
\pmmodified{2013-03-22 14:08:07}
\pmowner{jirka}{4157}
\pmmodifier{jirka}{4157}
\pmtitle{Rado's theorem}
\pmrecord{8}{35549}
\pmprivacy{1}
\pmauthor{jirka}{4157}
\pmtype{Theorem}
\pmcomment{trigger rebuild}
\pmclassification{msc}{31A05}
\pmrelated{HarmonicFunction}
\pmrelated{PerronFamily}

\endmetadata

% this is the default PlanetMath preamble.  as your knowledge
% of TeX increases, you will probably want to edit this, but
% it should be fine as is for beginners.

% almost certainly you want these
\usepackage{amssymb}
\usepackage{amsmath}
\usepackage{amsfonts}

% used for TeXing text within eps files
%\usepackage{psfrag}
% need this for including graphics (\includegraphics)
%\usepackage{graphicx}
% for neatly defining theorems and propositions
\usepackage{amsthm}
% making logically defined graphics
%%%\usepackage{xypic}

% there are many more packages, add them here as you need them

% define commands here
\theoremstyle{theorem}
\newtheorem*{thm}{Theorem}
\newtheorem*{lemma}{Lemma}
\newtheorem*{conj}{Conjecture}
\newtheorem*{cor}{Corollary}
\newtheorem*{example}{Example}
\newtheorem*{prop}{Proposition}
\theoremstyle{definition}
\newtheorem*{defn}{Definition}
\begin{document}
\begin{thm}[Rado]
Suppose $\Omega \subset {\mathbb{R}}^2$ is a \PMlinkname{convex}{ConvexSet} \PMlinkname{domain}{Domain2} with a smooth boundary $\partial \Omega$ and suppose that ${\mathbb{D}}$ is the unit disc.  Then given any homeomorphism $\mu : \partial {\mathbb{D}} \rightarrow \partial \Omega$, there exists a unique harmonic function $u : {\mathbb{D}} \rightarrow \Omega$ such that $u = \mu$ on $\partial {\mathbb{D}}$ and $u$ is a diffeomorphism.
\end{thm}

\begin{thebibliography}{9}
\bibitem{schoenyau}
R.\@ Schoen, S.\@ T.\@ Yau.  \emph{\PMlinkescapetext{Lectures on Harmonic
Maps}}.  International Press, Inc., Boston, Massachusetts, 1997
\end{thebibliography}
%%%%%
%%%%%
\end{document}
