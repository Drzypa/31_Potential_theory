\documentclass[12pt]{article}
\usepackage{pmmeta}
\pmcanonicalname{PerronFamily}
\pmcreated{2013-03-22 14:19:42}
\pmmodified{2013-03-22 14:19:42}
\pmowner{jirka}{4157}
\pmmodifier{jirka}{4157}
\pmtitle{Perron family}
\pmrecord{6}{35797}
\pmprivacy{1}
\pmauthor{jirka}{4157}
\pmtype{Definition}
\pmcomment{trigger rebuild}
\pmclassification{msc}{31B05}
\pmrelated{RadosTheorem}
\pmdefines{Perron function}

\endmetadata

% this is the default PlanetMath preamble.  as your knowledge
% of TeX increases, you will probably want to edit this, but
% it should be fine as is for beginners.

% almost certainly you want these
\usepackage{amssymb}
\usepackage{amsmath}
\usepackage{amsfonts}

% used for TeXing text within eps files
%\usepackage{psfrag}
% need this for including graphics (\includegraphics)
%\usepackage{graphicx}
% for neatly defining theorems and propositions
\usepackage{amsthm}
% making logically defined graphics
%%%\usepackage{xypic}

% there are many more packages, add them here as you need them

% define commands here
\theoremstyle{theorem}
\newtheorem*{thm}{Theorem}
\newtheorem*{lemma}{Lemma}
\newtheorem*{conj}{Conjecture}
\newtheorem*{cor}{Corollary}
\newtheorem*{example}{Example}
\theoremstyle{definition}
\newtheorem*{defn}{Definition}
\begin{document}
\begin{defn}
Let $G \subset {\mathbb{C}}$ be a region, $\partial_\infty G$
the extended boundary of $G$ and $S(G)$ the set of subharmonic functions
on $G$, then
if $f \colon \partial_\infty G \to {\mathbb{R}}$ is a continuous
function then the set 
\begin{equation*}
{\mathcal{P}}(f,G) := \{ \varphi : \varphi \in S(G) \text{ and }
\limsup_{z \to a} \varphi(z) \leq f(a) \text{ for all $a \in \partial_\infty G$} \} ,
\end{equation*}
is called the {\em Perron family}.
\end{defn}

One thing to note is the ${\mathcal{P}}(f,G)$ is never empty.  This is because
$f$ is continuous on $\partial_\infty G$ it attains a maximum, say $|f| < M$,
then the function $\varphi(z) := -M$ is in ${\mathcal{P}}(f,G)$.

\begin{defn}
Let $G \subset {\mathbb{C}}$ be a region and
$f \colon \partial_\infty G \to {\mathbb{R}}$ be a continuous
function then the function $u \colon G \to {\mathbb{R}}$
\begin{equation*}
u(z) := \sup \{ \phi : \phi \in {\mathcal{P}}(f,G) \} ,
\end{equation*}
is called the {\em Perron function} associated with $f$.
\end{defn}

Here is the reason for all these definitions.

\begin{thm}
Let $G \subset {\mathbb{C}}$ be a region and suppose $f \colon
\partial_\infty G \to {\mathbb{R}}$ is a continuous function.
If $u \colon G \to {\mathbb{R}}$ is the Perron function associated
with $f$, then $u$ is a harmonic function.
\end{thm}

Compare this with \PMlinkname{Rado's theorem}{RadosTheorem} which works with
harmonic functions with range in ${\mathbb{R}}^2$, but also gives a much
stronger statement.

\begin{thebibliography}{9}
\bibitem{Conway:complexI}
John~B. Conway.
{\em \PMlinkescapetext{Functions of One Complex Variable I}}.
Springer-Verlag, New York, New York, 1978.
\end{thebibliography}
%%%%%
%%%%%
\end{document}
