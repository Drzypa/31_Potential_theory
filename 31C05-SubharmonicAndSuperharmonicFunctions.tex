\documentclass[12pt]{article}
\usepackage{pmmeta}
\pmcanonicalname{SubharmonicAndSuperharmonicFunctions}
\pmcreated{2013-03-22 14:19:39}
\pmmodified{2013-03-22 14:19:39}
\pmowner{jirka}{4157}
\pmmodifier{jirka}{4157}
\pmtitle{subharmonic and superharmonic functions}
\pmrecord{12}{35796}
\pmprivacy{1}
\pmauthor{jirka}{4157}
\pmtype{Definition}
\pmcomment{trigger rebuild}
\pmclassification{msc}{31C05}
\pmclassification{msc}{31A05}
\pmclassification{msc}{31B05}
\pmrelated{HarmonicFunction}
\pmdefines{subharmonic}
\pmdefines{subharmonic function}
\pmdefines{superharmonic}
\pmdefines{superharmonic function}

\endmetadata

% this is the default PlanetMath preamble.  as your knowledge
% of TeX increases, you will probably want to edit this, but
% it should be fine as is for beginners.

% almost certainly you want these
\usepackage{amssymb}
\usepackage{amsmath}
\usepackage{amsfonts}

% used for TeXing text within eps files
%\usepackage{psfrag}
% need this for including graphics (\includegraphics)
%\usepackage{graphicx}
% for neatly defining theorems and propositions
\usepackage{amsthm}
% making logically defined graphics
%%%\usepackage{xypic}

% there are many more packages, add them here as you need them

% define commands here
\theoremstyle{theorem}
\newtheorem*{thm}{Theorem}
\newtheorem*{lemma}{Lemma}
\newtheorem*{conj}{Conjecture}
\newtheorem*{cor}{Corollary}
\newtheorem*{example}{Example}
\newtheorem*{prop}{Proposition}
\theoremstyle{definition}
\newtheorem*{defn}{Definition}
\theoremstyle{remark}
\newtheorem*{rmk}{Remark}
\begin{document}
First let's look at the most general definition.

\begin{defn}
Let $G \subset {\mathbb{R}}^n$ and let $\varphi \colon G \to
{\mathbb{R}} \cup \{ - \infty \}$ be an upper semi-continuous function,
then $\varphi$ is {\em subharmonic} if for every $x \in G$ and $r > 0$ such that
$\overline{B(x,r)} \subset G$ (the closure of the open ball of radius $r$ around $x$ is still in $G$) and every real valued continuous function $h$ on
$\overline{B(x,r)}$ that is harmonic in $B(x,r)$ and satisfies $\varphi(x) \leq h(x)$
for all $x \in \partial B(x,r)$ (boundary of $B(x,r)$) we have that
$\varphi(x) \leq h(x)$ holds for all $x \in B(x,r)$.
\end{defn}

Note that by the above, the function which is identically $- \infty$ is subharmonic, but some authors exclude this function by definition.
We can define {\em superharmonic} functions in a similar fashion to get that $\varphi$ is superharmonic if and only if $-\varphi$ is subharmonic.

If we restrict our domain to the complex plane we can get the following definition.

\begin{defn}
Let $G \subset {\mathbb{C}}$ be a region and let $\varphi \colon
G \to {\mathbb{R}}$ be a continuous function.  $\varphi$ is said to
be {\em subharmonic} if whenever $D(z,r) \subset G$ (where $D(z,r)$ is a closed
disc around $z$ of radius $r$) we have
\begin{equation*}
\varphi(z) \leq \frac{1}{2\pi} \int_0^{2\pi} \varphi(z+ r e^{i\theta}) d\theta ,
\end{equation*}
and $\varphi$ is said to be {\em superharmonic} if whenever $D(z,r) \subset G$
we have
\begin{equation*}
\varphi(z) \geq \frac{1}{2\pi} \int_0^{2\pi} \varphi(z+ r e^{i\theta}) d\theta .
\end{equation*}
\end{defn}

Intuitively what this means is that a subharmonic function is at any point
no greater than the average of the values in a circle around that point.  This implies that a non-constant subharmonic function does not achieve its maximum
in a region $G$ (it would achieve it at the boundary if it is continuous there).  Similarly for a superharmonic
function, but then a non-constant superharmonic function does not achieve its
minumum in $G$.  It is also easy to see that $\varphi$ is subharmonic if and only if $-\varphi$ is superharmonic.

Note that when equality always holds in the above equation then $\varphi$ would
in fact be a harmonic function.  That is, when $\varphi$ is both subharmonic and
superharmonic, then $\varphi$ is harmonic.

It is possible to relax the continuity statement above to take $\varphi$ only upper semi-continuous in the subharmonic case and lower semi-continuous in the
superharmonic case.  The integral will then however need to be the 
\PMlinkname{Lebesgue
integral}{Integral2} rather than the Riemann integral which may not be defined for such
a function.  Another thing to note here is that we may take ${\mathbb{R}}^2$ instead of ${\mathbb{C}}$ since we never did use complex multiplication.  In that case however we must rewrite the expression $z + r e^{i\theta}$ in \PMlinkescapetext{terms} of the
real and imaginary parts to get an expression in ${\mathbb{R}}^2$.

It is also possible generalize the range of the functions as well.  A subharmonic function could have a range of ${\mathbb{R}} \cup \{ -\infty \}$
and a superharmonic function could have a range of ${\mathbb{R}} \cup \{ \infty \}$.  With this generalization, if $f$ is a holomorphic function
then $\varphi(z) := \log \lvert f(z) \rvert$ is a subharmonic function if we 
define the value of $\varphi(z)$ at the zeros of $f$ as $-\infty$. 
Again it is important to note that with this
generalization we again must use the Lebesgue integral.

\begin{thebibliography}{9}
\bibitem{Conway:complexI}
John~B. Conway.
{\em \PMlinkescapetext{Functions of One Complex Variable I}}.
Springer-Verlag, New York, New York, 1978.
\bibitem{Krantz:several}
Steven~G.\@ Krantz.
{\em \PMlinkescapetext{Function Theory of Several Complex Variables}},
AMS Chelsea Publishing, Providence, Rhode Island, 1992.
\end{thebibliography}
%%%%%
%%%%%
\end{document}
