\documentclass[12pt]{article}
\usepackage{pmmeta}
\pmcanonicalname{BiharmonicEquation}
\pmcreated{2013-03-22 16:03:19}
\pmmodified{2013-03-22 16:03:19}
\pmowner{perucho}{2192}
\pmmodifier{perucho}{2192}
\pmtitle{biharmonic equation}
\pmrecord{9}{38107}
\pmprivacy{1}
\pmauthor{perucho}{2192}
\pmtype{Definition}
\pmcomment{trigger rebuild}
\pmclassification{msc}{31B05}

% this is the default PlanetMath preamble.  as your knowledge
% of TeX increases, you will probably want to edit this, but
% it should be fine as is for beginners.

% almost certainly you want these
\usepackage{amssymb}
\usepackage{amsmath}
\usepackage{amsfonts}
\usepackage{amsthm}

% used for TeXing text within eps files
%\usepackage{psfrag}
% need this for including graphics (\includegraphics)
%\usepackage{graphicx}
% for neatly defining theorems and propositions
%\usepackage{amsthm}
% making logically defined graphics
%%%\usepackage{xypic}

% there are many more packages, add them here as you need them

% define commands here
\newtheorem{definition*}{Definition.}
\begin{document}
\begin{definition*} A real-valued function $V\colon\mathbb{R}^n\to\mathbb{R}$ of \PMlinkname{class}{http://planetmath.org/encyclopedia/Cn.html} $C^4$, and satisfying the equation 
\begin{align}
\nabla^4 V=0,
\end{align}
also defines a biharmonic function, and (1) is called the biharmonic equation. Biharmonic operator is defined as 
$$\nabla^4:=\sum_{k=1}^n\frac{\partial^4}{\partial{x_k}^4}+
2\sum_{k=1}^{n-1}\sum_{l=k+1}^n\frac{\partial^4}{\partial{x_k}^2\partial{x_l}^2}\cdot$$
\end{definition*}
%%%%%
%%%%%
\end{document}
