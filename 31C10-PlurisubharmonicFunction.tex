\documentclass[12pt]{article}
\usepackage{pmmeta}
\pmcanonicalname{PlurisubharmonicFunction}
\pmcreated{2013-03-22 14:29:09}
\pmmodified{2013-03-22 14:29:09}
\pmowner{jirka}{4157}
\pmmodifier{jirka}{4157}
\pmtitle{plurisubharmonic function}
\pmrecord{9}{36019}
\pmprivacy{1}
\pmauthor{jirka}{4157}
\pmtype{Definition}
\pmcomment{trigger rebuild}
\pmclassification{msc}{31C10}
\pmclassification{msc}{32U05}
\pmsynonym{plurisubharmonic}{PlurisubharmonicFunction}
\pmdefines{plurisuperharmonic function}
\pmdefines{pseudoconvex function}

% this is the default PlanetMath preamble.  as your knowledge
% of TeX increases, you will probably want to edit this, but
% it should be fine as is for beginners.

% almost certainly you want these
\usepackage{amssymb}
\usepackage{amsmath}
\usepackage{amsfonts}

% used for TeXing text within eps files
%\usepackage{psfrag}
% need this for including graphics (\includegraphics)
%\usepackage{graphicx}
% for neatly defining theorems and propositions
\usepackage{amsthm}
% making logically defined graphics
%%%\usepackage{xypic}

% there are many more packages, add them here as you need them

% define commands here
\theoremstyle{theorem}
\newtheorem*{thm}{Theorem}
\newtheorem*{lemma}{Lemma}
\newtheorem*{conj}{Conjecture}
\newtheorem*{cor}{Corollary}
\newtheorem*{example}{Example}
\newtheorem*{prop}{Proposition}
\theoremstyle{definition}
\newtheorem*{defn}{Definition}
\begin{document}
\begin{defn}
Let $f \colon G \subset {\mathbb{C}}^n \to {\mathbb{R}}$ be an upper semi-continuous function.   $f$ is called {\em plurisubharmonic}
if for every complex line $\{ a + b z \mid z \in {\mathbb{C}} \}$
the function $z \mapsto f(a + bz)$ is a subharmonic function on the set
$\{ z \in {\mathbb{C}} \mid a + b z \in G \}$.
\end{defn}

Similarly, we could also define a {\em plurisuperharmonic} function just like
we have a superharmonic function, but again it just means that $-f$ is
plurisubharmonic, and so this extra \PMlinkescapetext{term} is not very useful.

\begin{defn}
A continuous plurisubharmonic function is said to be a {\em pseudoconvex function}.
\end{defn}

Note that since plurisubharmonic is a long word, many authors abbreviate
with {\em psh}, {\em plsh}, or {\em plush}.

\begin{thebibliography}{9}
\bibitem{Krantz:several}
Steven~G.\@ Krantz.
{\em \PMlinkescapetext{Function Theory of Several Complex Variables}},
AMS Chelsea Publishing, Providence, Rhode Island, 1992.
\end{thebibliography}
%%%%%
%%%%%
\end{document}
