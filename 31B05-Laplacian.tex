\documentclass[12pt]{article}
\usepackage{pmmeta}
\pmcanonicalname{Laplacian}
\pmcreated{2013-03-22 12:43:48}
\pmmodified{2013-03-22 12:43:48}
\pmowner{matte}{1858}
\pmmodifier{matte}{1858}
\pmtitle{Laplacian}
\pmrecord{18}{33030}
\pmprivacy{1}
\pmauthor{matte}{1858}
\pmtype{Definition}
\pmcomment{trigger rebuild}
\pmclassification{msc}{31B05}
\pmclassification{msc}{31B15}
\pmrelated{DAlembertian}
\pmrelated{Codifferential}
\pmdefines{Laplace operator}

\endmetadata

% this is the default PlanetMath preamble.  as your knowledge
% of TeX increases, you will probably want to edit this, but
% it should be fine as is for beginners.

% almost certainly you want these
\usepackage{amssymb}
\usepackage{amsmath}
\usepackage{amsfonts}

% used for TeXing text within eps files
%\usepackage{psfrag}
% need this for including graphics (\includegraphics)
%\usepackage{graphicx}
% for neatly defining theorems and propositions
%\usepackage{amsthm}
% making logically defined graphics
%%%\usepackage{xypic}

% there are many more packages, add them here as you need them

% define commands here

\newcommand{\sR}[0]{\mathbb{R}}
\newcommand{\sC}[0]{\mathbb{C}}
\newcommand{\sN}[0]{\mathbb{N}}
\newcommand{\sZ}[0]{\mathbb{Z}}

 \usepackage{bbm}
 \newcommand{\Z}{\mathbbmss{Z}}
 \newcommand{\C}{\mathbbmss{C}}
 \newcommand{\R}{\mathbbmss{R}}
 \newcommand{\Q}{\mathbbmss{Q}}



\newcommand*{\norm}[1]{\lVert #1 \rVert}
\newcommand*{\abs}[1]{| #1 |}
\begin{document}
Let $(x_1, \ldots, x_n)$ be Cartesian coordinates for some open set  $\Omega$ 
in $\sR^n$. 
Then the \emph{Laplacian} differential operator $\Delta$ is defined as
$$
\Delta = \frac{\partial^2 }{\partial x_1^2} + \cdots + \frac{\partial^2 }{\partial x_n^2}.
$$
In other words, if $f$ is a twice differentiable function $f:\Omega\to \sC$, then 
$$
\Delta f = \frac{\partial^2 f}{\partial x_1^2} + \cdots + \frac{\partial^2 f}{\partial x_n^2}.
$$
A coordinate independent definition of the Laplacian 
is $\Delta = \nabla \cdot \nabla$, i.e., $\Delta$ is the composition of 
gradient and codifferential.

A harmonic function is one for which the Laplacian vanishes.


\subsubsection*{Notes}
An older symbol for the Laplacian is $\nabla^2$ -- conceptually the scalar product of $\nabla$ with itself. This form is more favoured by physicists.

\subsubsection*{Derivation}
\htmladdnormallink{Click here}{<http://planetmath.org/?method=l2h&from=collab&id=76&op=getobj">} to see an article that derives the Laplacian in spherical coordinates.

%%%%%
%%%%%
\end{document}
